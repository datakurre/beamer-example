\documentclass[aspectratio=169]{beamer}

% Input
\usepackage[utf8]{inputenc}
\usepackage[T1]{fontenc}
\usepackage{lmodern,charter}
\usepackage[letterspace=100]{microtype}
\usepackage{textcomp}
\usepackage{upquote}

% Beamer
\usepackage{beamercolorthemedove}
\usepackage{beamerinnerthemecircles}
\setbeamertemplate{navigation symbols}{}

% Tikz
\definecolor{burlywood}{cmyk}{1,1,1,1}
\usepackage{tikz}
\usetikzlibrary{arrows,calc,fit,positioning,shapes,chains}

% Graphics
\usepackage{graphicx}
\usepackage{epstopdf}
\DeclareGraphicsExtensions{.png,.pdf,.eps}

% Various
\usepackage[outline]{contour}
\contourlength{1.75pt}
\usepackage{enumitem}
\usepackage{hyperref}
\usepackage{minted}

% Title
\title{Presentation Template}
\author{Asko Soukka <asko.soukka@iki.fi>}

% Begin
\begin{document}

\section{Title}

% Title
{
\usebackgroundtemplate{\includegraphics[trim=0 0 0 0,clip,width=0.5\paperwidth]{images/cover.jpg}}
\begin{frame}[plain,t]
  \begin{columns}[onlytextwidth]
    \begin{column}{0.45\textwidth}
    \end{column}
    \begin{column}{0.45\textwidth}
      \vspace{1.0cm}
      \par
      \centering
      \href{https://iki.fi/asko.soukka/}{Asko Soukka}
      \vspace{1.0cm}
      \par
      \Huge
      \bfseries
      Presentation Template
      \par
      \large
      \mdseries
      \vspace{1.0cm}\par
      \href{https://www.jyu.fi/}{\includegraphics[width=4cm]{images/logo.eps}}
    \end{column}
  \end{columns}
\end{frame}
}

\section{Part I}

{
\usebackgroundtemplate{\includegraphics[trim=0 0 0 0,clip,width=\paperwidth]{images/interlude-01.jpg}}%
\begin{frame}[plain,t]
  \setbeamercolor{background canvas}{bg=black}
  \vfill
  \centering
  \Huge
  \bfseries
  \lsstyle
  \contour{white}{Hello there!}
\end{frame}
}

\begin{frame}[plain,t]
  \vspace{1.0cm}\par
  \Huge
  \bfseries
  \centering Just a bunch of bullets
  \huge
  \vspace{1.0cm}\par
  \mdseries
  \begin{itemize}[label=--]
    \item Functional DSL for building stuff
    \item Builds named by their expression hash
    \item Builds isolated and reproducible
    \item Build results immutable and shareable
  \end{itemize}
\end{frame}

\section{Part II}

{
\usebackgroundtemplate{\includegraphics[trim=0 0 0 0,clip,width=\paperwidth]{images/interlude-02.jpg}}%
\begin{frame}[plain,b]
  \setbeamercolor{background canvas}{bg=black}
  \Huge
  \bfseries
  \lsstyle
  \contour{white}{What's up?}
  \vspace{1.0cm}
\end{frame}
}

\begin{frame}[plain]
\begin{center}
\begin{tikzpicture}[
    start chain=going right,
    diagram item/.style={
        on chain,
        join
    }
]
\node[
    diagram item,
    label=center:Internet
]{\includegraphics{icons/cloud}};

\node[
    diagram item,
    label=right:Cable Modem
]{\includegraphics{"icons/cable modem"}};

\node[
    continue chain=going below,
    diagram item,
    label=right:Workgroup Switch
]{\includegraphics{"icons/workgroup switch"}};

\node[
    start branch=1 going below right,
    diagram item,
    label=below:PC
]{\includegraphics{icons/pc}};

\node[
    start branch=2 going below left,
    diagram item,
    label=below:Supercomputer
]{\includegraphics{icons/supercomputer}};

\node[
    diagram item,
    label=below:Office
]{\includegraphics{"icons/end office "}};
\end{tikzpicture}
\end{center}
\end{frame}

\section{Part III}

{
\usebackgroundtemplate{\includegraphics[trim=0 0 0 0,clip,width=\paperwidth]{images/interlude-03.jpg}}%
\begin{frame}[plain,t]
  \setbeamercolor{background canvas}{bg=black}
  \vfill
  \centering
  \Huge
  \bfseries
  \lsstyle
  \contour{white}{Isn't that cute?}
\end{frame}
}

\begin{frame}[plain,fragile]
  \vspace{1.0cm}
  \Huge
  \bfseries
  \centering Dockerfile
  \huge
  \vspace{1.0cm}
  \mdseries
\begin{minted}[breaklines]{Dockerfile}
FROM scratch
ADD app.tar.gz /
EXPOSE 8080
ENTRYPOINT ["/bin/app"]
\end{minted}
\end{frame}

\section{Conclusions}

{
\usebackgroundtemplate{\includegraphics[trim=0 0 0 0,clip,width=\paperwidth]{images/interlude-04.jpg}}%
\begin{frame}[plain,t]
  \setbeamercolor{background canvas}{bg=black}
  \vfill
  \centering
  \Huge
  \bfseries
  \lsstyle
  \contour{white}{Questions?}
\end{frame}
}

\end{document}
